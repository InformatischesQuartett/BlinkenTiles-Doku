\clearpage

\mysubsection{Meike Zöckler}{Werbebotschaft – Einsatz, Sinn und Zweck}

Eine durch die Traversen und Traversenstative mobile Version der Installation ließe sich für verschiedene Einsatzzwecke verwenden, vorausgesetzt, der Raum, in dem sie aufgebaut werden soll, ist hoch genug.

Im vorliegenden Fall dient sie der Bewerbung der interdisziplinären Koorperation zwischen der Hochschule Furtwangen und der Musikhochschule Trossingen. Viele wissen nichts von dieser Kooperation und nutzen demnach auch nicht das Angebot, dort Wahlpflichtveranstaltungen mit Schwerpunkt Musik zu belegen. Umgekehrt genauso, dass nur wenige Studierende das Angebot nutzen, Veranstaltungen in Furtwangen zu belegen.

Ziel ist es, eine gewisse Awareness für den Partnerstudiengang Musikdesign zu entwickeln und dadurch Kompetenzen im musikalischen Bereich zu erweitern, die künftig möglicherweise durch gemeinsame Projekte vertieft werden könnten.

Die künftige Kooperation könnte verschiedene Studiengänge umfassen:

\vspace{0.5em}

\begin{itemize}
	\item \textbf{Konzeption und Design:} Medienkonzeption (B.A.), Design interaktiver Medien (M.A.)
	\item \textbf{technische Umsetzung:} Medieninformatik (B.A.), Medieninformatik (M.A.)
	\item \textbf{Musikalischer Input:} Musikdesign (B.Mus.)
	\item \textbf{Onlinepräsentation und Marketing:} Online Medien (B.Sc.)
\end{itemize}

\vspace{0.5em}

Zur choreografischen Vorführung und Untermalung könnte die Kooperation auch auf \emph{Sing \& Move} mit Schwerpunkt \emph{Music \& Movement (B.Mus.)} erweitert werden.

BlinkenTiles bietet von allen Studiengängen das Beste: Ansprechendes Design, gute technische Umsetzung, Spiel mit Musik und Bewegung, in dem sich die Nutzer später frei entfalten können, um gemeinsam Musik zu erzeugen.

\clearpage






