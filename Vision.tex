\mysubsection{Meike Zöckler}{Vision}

Ein großer Mehrwert würde durch den Verzicht auf Beamer und die Traversen entstehen. So könnte man die Installation an beliebigen Orten,zum Beispiel im hellen Sonnenlicht oder in niedrigen Räumen, einsetzen. Optimal wäre für diese Art der Umsetzung eine Art Tanzmatte oder leuchtende Bodenplatten im Stile der 80er Jahre. So könnten außerdem technische Erfassungsprobleme der Kinect und die Schattenbildung durch den Beamer ausgeschlossen werden. Zusätzlich könnte in dieser Ausführung nicht nur auditives und visuelles, sondern auch haptisches Feedback durch Vibration der Platten eingeführt werden.\\

Der Challenge Mode ließe sich zudem weiter ausbauen. Je nach Anzahl der Spieler und nach Schwierigkeitsgrad könnte es vier verschiedene Level oder Niveaus geben: Easy, Medium, Expert und Boss. Für jede dieser Stufen sollte ein ausreichend großer Pool an Songs zur Verfügung stehen, die per Zufall ausgewählt werden, um genügend Abwechslung zu garantieren. Die Frage ist, wie die Auswahl und der Input gestaltet werden. Wird es eine Konsole geben, erkennt der Boden selbstständig, wie viele Menschen sich darauf befinden und wählt selbstständig ein Level aus?\\

Alternativ wäre auch eine Challenge im Sinne diverser Tanzspiele wie DDR (Dance Dance Revolution aus dem Hause Konami). Man teilt die Felder in Bereiche oder Zonen auf und die Spieler müssen gegeneinander antreten und auf die vorgegebenen Felder springen. Je nach dem, wie richtig oder falsch, wie schnell man die Felder erreicht hat, bekommt man Punkte. Dazu allerdings wäre außerdem eine Anzeigetafel oder ein Screen vonnöten, um den Punktestand darzustellen.