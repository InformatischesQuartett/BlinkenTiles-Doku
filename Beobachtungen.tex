\mysubsection{Meike Zöckler}{Nutzergruppen-Analyse}

Am Tag der Medien konnten unterschiedliche Nutzertypen bzw. Verhaltensweisen im Umgang mit der Installation beobachtet werden. Diese werden folgendermaßen kategorisiert:\\

\begin{itemize}
	\item \textbf{Kategorie 1:} Laufen über die Projektion ohne sie eines Blickes zu würdigen.
	\item \textbf{Kategorie 2:} Laufen skeptisch an der Projektion vorbei; wollen nicht im Mittelpunkt stehen oder trauen der Sache nicht.
	\item \textbf{Kategorie 3:} Steht am Rand und diskutiert mit anderen Leuten, was es ist, wie es funktioniert und ob sie das nicht doch mal ausprobieren sollten...?
	\item \textbf{Kategorie 4:} Leute, die es zögerlich ausprobieren und sich dann doch etwas damit beschäftigen, bevor sie weiterhin skeptisch weitergehen
	\item \textbf{Kategorie 5:} Leute, die sich direkt darauf einlassen und dann einige Zeit darauf verbringen und alle Möglichkeiten austesten
	\item \textbf{Kategorie 6:} Leute, die damit spielen; sowohl Challenge als auch Spiel im Freestyle, indem sie mit der Erkennung spielen und Muster laufen oder die versuchen, mit dem Impuls mitzulaufen, die Himmel und Hölle hüpfen, die sich auf den Boden werfen, um möglichst viele Felder gleichzeitig abzudecken, die balancieren und den ganzen Körper verbiegen um neue Muster auszuprobieren, die Ausdruckstanz wagen, die es als Meditation wahrnehmen (besonders mit der sphärischen Standardversion)
\end{itemize}

Neben der direkten Interaktion mit BlinkenTiles wurden außerdem Fotos aus sämtlichen Perspektiven geschossen, es wurde gefilmt, diese Fotos und Videos geteilt, untereinander und teilweise über Facebook. Für den Challenge Mode wurden aktiv alle umstehenden Personen angesprochen und zum Mitmachen animiert. Dabei spielte es keine Rolle, ob sie sich kannten oder nicht; die Gruppen setzten sich zusammen aus Nutzern jeden Alters, darunter Studierende, Professoren, Mitarbeiter, Externe wie Besucher. Und sehr viele haben sich für die Technik dahinter interessiert und sich diese ausführlich erklären lassen.
