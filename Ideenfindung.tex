\mysubsection{Meike Zöckler}{Ideenfindung}

Am Anfang war eine Idee. Oder auch zwei.

\subsubsection{Idee 1: Interaktiver Werbescreen}

Die erste davon sah ein interaktives Game vor. Unliebsame Wartesituationen sollen endlich sinnvoll genutzt werden: mit Spiel und Spaß im Multiplayer statt gähnender Langeweile. An typischen \enquote{Warteorten} sollten Screens aufgestellt werden. Per QR-Code sollte sich der Wartende in das Spiel einloggen. Zur Verfügung stehen sollten bekannte Arcade-Klassiker wie bspw. Pong oder Pacman. Jeder hinzukommende Spieler übernimmt dabei eine Spielfigur. Einer ist Pacman, die nächsten werden die Geister, die Pacman jagen sollen. Je mehr Spieler dazu kommen, desto größer wird das Labyrinth, desto mehr Geister und Pacmans gibt es. Das eigene Smartphone dient dabei als Controller.

Die \textbf{Schwierigkeiten} bei dieser Idee umfassen mehrere Aspekte. Zum einen ist die Hürde zur Teilnahme deutlich höher als bei der nachfolgend beschriebenen Idee. Der Wartende braucht ein Smartphone. Er muss es gut genug bedienen können, um sich via QR-Code einzuloggen und um es dann auch noch als Controller verwenden zu können. Um dem entgegen zu kommen, sollte eine möglichst einfache, übersichtliche und für verschiedene Spiele einsetzbare Steuerung implementiert werden. Ein weiteres Problem hätte die Konnektivität darstellen können. Für ein einwandfreies Spielerleben sind eine stabile und schnelle Verbindung sowie ein unkompliziertes Einloggen vonnöten.

Von technischen Schwierigkeiten abgesehen, benötigt der potentielle Spieler genügend (Warte-)Zeit, um sich darauf einzulassen und mit dem nötigen Spaß und Interesse an das Spiel heranzugehen. Er sollte dieses Erlebnis eben mit Spaß verknüpfen und nicht mit Stress und Zeitdruck, was an Wartestationen wie Bushaltestellen oder Bahnhöfen der Fall sein könnte, wo er mitunter auch gewaltsam aus seinem Spielerleben gerissen wird, sollten Bus oder Bahn endlich einfahren. Außerdem hätte der Call-to-Action ein sehr deutlicher und ansprechender sein müssen, um neben den Hardcore-Gamern auch Casual- oder Nongamer anzusprechen. Hier hätte man ggf. einen entsprechenden Anreiz in Form von Gewinnchancen schaffen müssen.

\subsubsection{Idee 2: Der Entschleuniger}

Die zweite Idee beinhaltete ein Klavier. Dieses Klavier sollte auf einem Platz stehen. Sobald Passanten vorbeilaufen, wird Musik abgespielt. Wobei diese Musik in Tempo des Laufschritts und die Tonhöhe oder das Instrument durch die Statur der Person angepasst sein sollte. Überlegt wurden zwei verschiedene Modi:

\textbf{Modus 1:} Man sollte gemeinsam Musik machen können. Man sollte mit dem eigenen zugewiesenen Instrument und denen anderer Menschen ein Lied \enquote{komponieren}. Der Spaß an der Musik und am Experimentieren sollte im Vordergrund stehen, wobei sich hier die Frage aufdrängte, ob denn aus dieser einfachen Spielerei auch ein guter Song werden könnte. Die Installation könnte schnell an Attraktivität verlieren, sollten die erzeugten Klänge und Melodien nicht gut klingen.

\textbf{Modus 2:} Dieser Modus verfolgte ein klares Ziel: Entschleunigung. Zu hören ist zunächst ein einfacher Song. Läuft jemand vorbei und das viel zu schnell und zu hektisch, passt sich der Song dem Tempo des Passanten an. Läuft er zu eilig, ist auch das Lied viel zu schnell. Erst wenn er sein Tempo zurücknimmt, kann er das Lied im Originaltempo hören. Damit sollte Aufmerksamkeit auf die Schönheit der einfachen Dinge und eines entschleunigten, vielleicht sogar entspannteren Lebens gelenkt werden. Überlegt wurde, die Installation in dieser Variante auch als Werbemöglichkeit für (Luxus-)Uhren einzusetzen.

Zwischenzeitlich wurde überlegt, zur Verdeutlichung des zu schnellen Laufens die \enquote{Temposünder} zu blitzen, was aber verworfen wurde, da die Passanten von selbst auf ihr zu hohes Tempo aufmerksam werden sollten.

Bei dieser Idee waren die \textbf{Schwierigkeiten} anderer Natur als zuvor bei der Interaktiven Werbeplattform. Einerseits hätte man zur Präsentation einen großen Platz oder zumindest eine größere Fläche mit hoher Passantenzahl benötigt. Somit eignet sich die Installation nicht für alle Orte. Andererseits hätten sich auch hier die Menschen erst darauf einlassen müssen. Der Reiz hätte stark genug sein müssen, damit sie erkennen, dass die Installation auf sie direkt reagiert. Als Idee zur Lösung wurde vorgeschlagen, dass parallel zu den echten Passanten Comicfiguren über eine bestehende Leinwand laufen und sie so neben dem akustischen auch einen optischen Reiz wahrnehmen, der sie zur Reaktion zwingt.

Da beide Ideen nicht zur vollständigen Zufriedenheit gereichten, ist aus deren Verschmelzung schließlich Idee Nummer drei \textbf{BlinkenTiles} entstanden.