\mysubsection{Sarah Häfele}{Praxiserfahrungen}\label{ssec:Praxis}

Nach mehrmaligem Testen wurde der Prototyp am Tag der Medien zum ersten Mal am Publikum getestet. Die Rückmeldung und der Anklang waren positiv und die Installation wurde gut frequentiert. Im ersten richtigen Betrieb fielen jedoch natürlich einige Mängel auf, die verbesserungswürdig wären.  
 
\subsubsection{Problem-Analyse}
Neben den schon von Fabian Gärtner in \autoref{ssec:problems} aufgezählten Problemen sollen hier weitere Problemfaktoren und Verbesserungsvorschläge skizziert werden.\\
Der Idle Modus, der startet, wenn keine Besucher auf dem Spielfeld sind, ist nicht eindeutig genug und sollte eventuell durch eine andere Farbe deutlicher gemacht werden. Auch der Challenge Modus hat Farbgebungsprobleme und zeigt so nicht prominent genug, wenn ein neues Lied beginnt. Hier könnte man sich auch dem Presentation-Screen bedienen und sogar akustische Signale verwenden. Ein allgemeines akustisches Feedback würde viele Signale einfacher erkennbar machen.\\
Die Impulslinie ist von oben gut erkennbar, steht man aber auf dem Spielfeld, ist sie schon schwerer zu sehen und man erkennt vor allem im Challenge Modus nicht richtig, welches Feld als nächstes belegt werden muss. Der Impuls muss besser dargestellt werden, zum Beispiel wie bei einem Ladebalken, der großflächig die Felder abdeckt, die nicht mehr aktuell sind. Dieses Symbol sollte allgemein bei der Zielgruppe bekannt sein und sollte den Impuls prominenter machen. Weitere Verbesserungsmöglichkeiten wären Notensymbole, die die nächst zu besetzenden Felder markieren.\\ 


Positiv: animiert zum mitmachen
gute position
animation der felder gut erkennbar
Freestyle modus lädt zur kreativität ein
challenge modus lief bei den geringeren schwieirgkeiten auch gut
\subsubsection{Der Tag der Medien - Impressionen}