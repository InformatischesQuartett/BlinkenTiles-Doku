\mysubsection{Sarah Häfele}{Probleme und Verbesserungen}\label{ssec:Praxis}

Nach mehrmaligem Testen wurde der Prototyp am Tag der Medien zum ersten Mal am Publikum getestet. Die Rückmeldung und der Anklang waren positiv und die Installation wurde gut frequentiert. Im ersten richtigen Betrieb fielen jedoch natürlich einige Mängel auf, die verbesserungswürdig sind.\\

Neben den schon von Fabian Gärtner in \autoref{ssec:problems} aufgezählten Problemen sollen hier weitere Problemfaktoren und Verbesserungsvorschläge skizziert werden.\\
Der Idle Modus, der startet, wenn keine Besucher auf dem Spielfeld sind, ist nicht eindeutig genug und sollte eventuell durch eine andere Farbe deutlicher gemacht werden. Auch der Challenge Modus hat Farbgebungsprobleme und zeigt so nicht prominent genug, wenn ein neues Lied beginnt. Hier könnte man sich auch dem Presentation-Screen bedienen und sogar akustische Signale verwenden. Ein allgemeines akustisches Feedback würde viele Signale einfacher erkennbar machen.\\
Die Impulslinie ist von oben gut erkennbar, steht man aber auf dem Spielfeld, ist sie schon schwerer zu sehen und man erkennt vor allem im Challenge Modus nicht richtig, welches Feld als nächstes belegt werden muss. Der Impuls muss besser dargestellt werden, zum Beispiel wie bei einem Ladebalken, der großflächig die Felder abdeckt, die nicht mehr aktuell sind. Dieses Symbol sollte allgemein bei der Zielgruppe bekannt sein und sollte den Impuls prominenter machen. Weitere Verbesserungsmöglichkeiten wären Notensymbole, die die nächst zu besetzenden Felder markieren.\\ 
Durch Windowsupdates just an diesem Tag kam es zu Treiberproblemen, weswegen die Kommunikation über das Internet nicht mehr richtig funktionierte, der Presenter-Screen deshalb schnell umgeschrieben werden musste und die DMX-Steuerung nicht mehr reagierte. Daraufhin wurden die Scheinwerfer vom System abgetrennt, um weitere Systemabstürze zu verhindern. Zukünftig gilt hier: ein stabiles System darf am Tag vor der Präsentation nicht mehr angefasst werden.\\
Zusätzlich zu den aufgetretenen Problemen wurden auch einige andere Verbesserungsvorschläge beim Beobachten der Probanden klar:\\
Der Wunsch nach einem noch kreativeren Weg, Musik zu machen, kam mehrmals auf. Hier wäre ein Modus ohne Impuls und mit freiem Rhythmus sicher spannend, würde aber natürlich ein wenig von dem eigentlichen Konzept abweichen. Auch bietet die Kinect gute Möglichkeiten für noch mehr Interaktion seitens der Nutzer. So könnte der Impuls mit den Bewegungen der Personen schneller oder die Lautstärke der einzelnen Töne durch ein Handsignal geregelt werden. Interessant sind auch Soundmatritzen, die verschiedene Pulse verwenden, wie z.B. sich zyklisch ausbreitende oder zufallsgesteuerte Impulse.\\
Bei all den Problemen war die Installation trotzdem ein großer Erfolg. Natürlich war dies auch der guten Position geschuldet, die die Installation gleich beim Betreten des I-Baus in den richtigen Fokus rückte. Zum anderen weckten Musik, Farben, Bewegung und die Herausforderung die Kreativität des doch meist medienaffinen Publikums. Da sich einige für die Installation interessierten, lockten diese auch weitere Personen an. Details wie die Animation der Felder und der Challenge Modus kamen gut an. Der Freestyle Modus, der keine großen Vorgaben hat, war jedoch der beliebteste Modus und wurde bis zum Abbau der Installation begeistert genutzt.