\mysubsection{Johannes Winter}{Zielgruppenanalyse}

Der Vorteil der Installation besteht darin, dass es (zumindest im experimentellen Modus) keine Barrieren für bestimmte Altersgruppen gibt. Menschen jeden Alters ist es möglich mit der Installation zu interagieren, da weder motorisch noch kognitiv anspruchsvolle Usereingaben von Nöten sind. Das Userinterface verlangt außerdem keine Vorkenntnisse, wie es zum Beispiel bei einer App der Fall wäre. Auch ist die Hemmschwelle sowohl technisch als auch psychologisch sehr niedrig, da auf Userseite keine Geräte synchronisiert werden und keine peinlichen Bewegungen vollzogen werden müssen. Auch sind für das effektive Benutzen der Installation keine musikalischen Kenntnisse gefordert.
Dennoch zielt die Installation und die damit verbundene Aussage auf junge, experimentierfreudige und möglicherweise auch musikalisch interessierte Menschen. Besonders der Challenge Modus, der einen deutlich bewegungsorientierteren Ansatz bietet und auch kognitiv anspruchsvoller ist, ist eher für Jugendliche und junge Erwachsene interessant, die Spaß an Bewegung und Herausforderungen in Kombination mit Musik haben.\\
Schwierigkeiten, Lösungen und schließlich das Endergebnis werden in den nachfolgenden Kapiteln zur Umsetzung beschrieben.