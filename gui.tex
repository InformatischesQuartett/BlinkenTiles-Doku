\subsection{Die grafische Oberfläche (GUI)}
\label{gui}

Teil der grafischen Oberfläche ist das Spielfeld (Abb. \ref{fig:gui-tiles}), mit dem die Nutzer wie im Konzept beschrieben, interagieren kann. Es stellt nicht nur die einzelnen Töne als quadratische Flächen dar, sondern dient auch als Informationsfeld (siehe auch Abschnitt \ref{ssec:CtA}). Die Tiles können ihre Farbe wechseln und haben verschiedene Animationen. Näheres hierzu in Abschnitt \ref{sec:tiles}.

\begin{figure}[htbp] 
  \centering
     \includegraphics[width=0.9\textwidth]{images/gui-tiles}
  \caption{Das Spielfeld}
  \label{fig:gui-tiles}
\end{figure}

Neben dem Spielfeld wurde für die Personenerkennung der BlinkenTiles-Anwendung eine weitere Bedienoberfläche entwickelt (siehe Abb. \ref{fig:blob}). Sie liegt auf dem linken Teil der gesamten GUI und kann so nur auf dem PC-Bildschirm eingesehen werden, während der rechte Bildschirm das Spielfeld darstellt und vom Beamer auf den Boden projiziert wird. Die Einstellungen ermöglichen u.\,a.  die Kalibrierung der Kinect und die Abstimmung mit der tatsächlichen Spielfeldgröße. Näheres hierzu findet sich in Abschnitt \ref{sec:objdet}.

Eine dritte grafische Oberfläche wurde für den Presentation-Screen entwickelt, siehe hierzu Abschnitt \ref{ssec:CtA}.

\begin{figure}[htbp] 
  \centering
     \includegraphics[width=0.9\textwidth]{images/Blob}
  \caption{GUI für die Bildverarbeitung}
  \label{fig:blob}
\end{figure}

\newpage