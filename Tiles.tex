\mysubsection{Alexander Scheurer}{Spielfeld und Felder}
\label{sec:tiles}

Das Spielfeld und die Felder werden von zwei Klassen bestimmt, die \emph{TileController}- und \emph{TileBehaviour}-Klasse. Erstere ist die zentrale Klasse. Sie beinhaltet die Spiellogik, baut das Spielfeld auf und verarbeitet Eingaben (von Personenerkennung, Maus, Tastatur, Konfiguration und Idle Mode).

Die Spiellogik umfasst überwiegend das Timing der Abläufe und in Abhängigkeit davon das Zuweisen von Zuständen an die einzelnen Felder. Die Zustände, die zugewiesen werden, beschreiben z.B. ob auf diesem eine Person steht, ob sich der Zeitindex derzeit in der Reihe dieses Feldes befindet oder ob sogar beides genannte zutrifft. Dies hätte zur Folge, dass auch der entsprechende Ton zu dem Feld abgespielt werden muss. Die \emph{TileController}-Klasse weist den Feldern den entsprechenden Zustand sequenziell zu:

\begin{enumerate}
\item Alle Felder werden zurückgesetzt (\emph{Clear pass})
\item Der Zustand für Zeitimpuls wird bei den entsprechenden Feldern gesetzt (\emph{Timer pass})
\item Der Zustand für Feldern auf denen Personen erkannt wurden, werden gesetzt (\emph{People pass})
\item Der vereinigte Zustand von 2. und 3. wird gesetzt (\emph{Shake'n'Play pass})
\end{enumerate}

Verarbeitet wird dieser Zustand von jedem Feld selbst mit Hilfe der \emph{TileBehaviour}-Klasse. In dieser passiert die Umsetzung von Zustand zu Aktion, also welcher Shader (Farbe) für das Feld gesetzt wird oder ob es wackeln soll, weil es den entsprechenden Zustand vom 4. Schritt zugewiesen bekommen hat.

Zum Start des Spiels hat der \emph{TileController} die Aufgabe das Spielfeld aufzubauen, dies macht er anhand von JSON-Konfigurationsdateien einmal für die Grundkonfiguration als auch Konfigurationen für jeden Song. Die Grundkonfiguration definiert unter anderem welche Ausmaße das Spielfeld hat, oder auch welche Farbe einem bestimmten Feldzustand zugewiesen ist. Die Song-Konfiguration beinhaltet zum Beispiel die Angabe über die Beats pro Minute und die Pfade zu den Audiodateien.

Benutzereingaben nimmt der \emph{TileController} entgegen und veranlasst die entsprechende Aktion. Zum Beispiel einen Songwechsel oder das verarbeiten eines Feldzustandes von der Personenerkennung oder dem Idle-Node.


