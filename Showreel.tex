\mysubsection{Johannes Winter}{Showreel}

Eine der Anforderungen der Veranstaltung an die Studierenden bestand darin, einen abschließenden Teaser zu erstellen, welcher einen kurzen Überblick über die Installation geben, Impressionen vermitteln und Lust auf die Interaktion mit der Installation machen soll. Das Showreel (siehe Abb. \ref{fig:teaser} (a)), geschnitten von Fabian Gärtner, Meike Zöckler und Johannes Winter, kann über YouTube\footnote{\url{https://www.youtube.com/watch?v=oIt5Z8fHog0}} und auf Vimeo\footnote{\url{http://vimeo.com/118690388}} angeschaut werden. Der QR-Code in Abb. \ref{fig:teaser} (b) führt direkt zum Video auf YouTube.

\begin{figure}[htbp]
\subfigure[Ausschnitt aus dem Showreel]{\includegraphics[width=0.59\textwidth]{images/showreel.png}}\hfill
\subfigure[QR-Code zum Teaser]{\includegraphics[width=0.39\textwidth]{images/qrcode.png}}
\caption{Der Teaser zur Installation}
\label{fig:teaser}
\end{figure}

Das Showreel soll für die Installation begeistern und Lust darauf machen diese selbst auszuprobieren. Dabei steht klar der spielerische, musikalische und gemeinschaftliche Aspekt im Vordergrund. Da die Installation musikalisch geprägt ist, wurde auf der Audioebene bewusst auf Sprachmitschnitte, Umgebungsgeräusche und Sprechertexte verzichtet. Es wurde besonderen Wert auf die Synchronisation der Musik und des Bildes gelegt, um eine bestmögliche Synthese der Sinneseindrücke zu erlangen.

Auch das dramaturgische Konzept des Showreels lehnt sich an dem musikalischen Aufbau des verwendeten Stückes an. So sind bei Build-Ups vorwiegend Timelapse Aufnahmen zu sehen, die dann an Punkten des musikalischen Klimax von emotionalen und von Reaktionen geprägten Aufnahmen gebrochen werden. Bei solchen Aufnahmen wird ein breites Spektrum an Userreaktionen und Interaktionsmöglichkeiten gezeigt. Dabei wurden verschiedene Umgangsmöglichkeiten mit der Installation in den Fokus gerückt, um die vielfältigen Möglichkeiten aufzuzeigen. Somit war es wichtig zu zeigen, welch unterschiedliche Formen der Joy-of-Use bei der Installation annimmt. Für einige Nutzer lag der Wettstreit mit anderen im Vordergrund, während andere mit der Erfassung verschiedener Gliedmaßen experimentierten. Wieder andere gingen sehr zögerlich und vorsichtig mit der Installation um und waren überrascht über die musikalischen Folgen ihrer Interaktion. Andere Nutzer interagierten eher unbewusst mit der Installation, indem sie einfach darüber liefen und on-the-fly interessante Tonfolgen kreierten, die weitere Zuschauer anlockten.