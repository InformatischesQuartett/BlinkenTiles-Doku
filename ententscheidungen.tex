\subsection{Entwicklungs-Entscheidungen}\label{ssec:entscheidungen}

Zur Entwicklung des Prototypen haben wir uns softwareseitig für die Gameengine Unity\footnote{http://www.unity3d.com} entschieden.
Gründe hierfür waren, dass der Umgang mit Unity den meisten Projektmitgliedern aus anderen Lehrveranstaltungen geläufig war und die Anbindung externer Softwarebibliotheken (z.B. OpenCV zur Personenerkennung, vgl. Seite \pageref{sec:objdet}) sehr einfach ist.\\
Für die Kommunikation mit den LED-Scheinwerfern wurde \emph{Freestyler}\footnote{http://www.freestylerdmx.be/}, eine kostenfreien DMX-Lichtsteuerungs-Software verwendet, da es für das verwendete DMX512-Interface in der Produktbeschreibung empfohlen wurde und einwandfrei kompatibel ist. Außerdem bietet Freestyler die Möglichkeit per \emph{SendMessage()} problemlos aus anderen Programmen (hier die \enquote{Blinken Tiles}-Anwendung) auf Funktionen zuzugreifen und und diese zu steuern.\\
Die wichtigsten Parameter werden über eine Konfigurationsdatei eingestellt. Dadurch können die einzelnen Parameter geändert werden ohne das komplette Projekt noch einmal kompilieren und exportieren zu müssen (siehe \autoref{sec:objdet}, \autoref{sec:tiles} und\autoref{ssec:DMX}).