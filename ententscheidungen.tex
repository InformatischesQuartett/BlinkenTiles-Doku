\mysubsection{Das Team}{Softwaretechnische Entscheidungen}
\label{ssec:entscheidungen}

Zur Entwicklung der Anwendung für den Prototypen fiel die Entscheidung auf die 3D-Game-Engine Unity\footnote{\url{http://www.unity3d.com}}. Gründe hierfür waren u.\,a., dass der Umgang mit Unity den meisten Projektmitgliedern aus anderen Lehrveranstaltungen geläufig war und die Anbindung externer Softwarebibliotheken (z.B. das \emph{Microsoft Kinect}-SDK oder \emph{OpenCV} zur Personenerkennung, vgl. Abschnitt \ref{sec:objdet}) über \CS{} als Programmiersprache vergleichsweise einfach ist. Zudem wurde entschieden Git als Sourcecode-Verwaltung einzusetzen, wodurch der Programmcode von BlinkenTiles jederzeit online abrufbar ist\footnote{\url{https://github.com/InformatischesQuartett/BlinkenTiles}}. Die wichtigsten Einstellungen in BlinkenTiles können über eine externe Konfigurationsdatei vorgenommen werden. Dadurch können alle Parameter, die in den folgenden Abschnitten näher erläutert werden (vgl. Abschnitte~\ref{ssec:DMX}, \ref{sec:tiles} und \ref{sec:objdet}), geändert werden, ohne dass die komplette Anwendung noch einmal kompiliert und exportiert werden muss.

Für die Kommunikation mit den LED-Scheinwerfern wurde \emph{Freestyler}\footnote{\url{http://www.freestylerdmx.be/}}, eine kostenfreie und umfangreiche DMX-Lichtsteuerungs-Software verwendet, da diese für das verwendete DMX512-Interface in der Produktbeschreibung empfohlen wurde und einwandfrei kompatibel ist. Außerdem bietet Freestyler die Möglichkeit per Befehl problemlos aus anderen Programmen (hier die BlinkenTiles-Anwendung) heraus auf Funktionen zuzugreifen, diese zu steuern und so die LED-Scheinwerfer anzusprechen. Mehr dazu im folgenden Abschnitt.

